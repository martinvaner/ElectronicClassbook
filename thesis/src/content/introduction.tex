\begin{introduction}
Třídní knihy jsou prostředkem pro zápis informací o proběhnutém vyučování, absencí žáků a dalších informací, které souvisí s výukou na základních a středních školách. Vedení třídní knihy je ze zákona povinné, protože, jako jediný školní dokument, obsahuje průkazné informace o poskytnutém vzdělání a jeho průběhu. Z tohoto důvodu je potřeba zajistit třídní knihu proti ztrátě a řádně archivovat podle zákona.

V současné době, kdy moderní technologie jsou již každodenní součástí života, je vedení třídní knihy v klasické papírové podobě přežitkem. Vše navíc umocňuje současná situace, kdy se společnost potýká s pandemií koronaviru. Dálková výuka komplikuje práci s papírovou třídní knihou, ať už se jedná o samotný zápis nebo o omlouvání absence. 

Tento problém lze vyřešit vytvořením informačního systému pro správu třídních knih. Informační systém v podobě webové aplikace je přístupný přes internet, což pomůže řešit problémy spojené s třídní knihou při dálkové výuce. Zároveň odpadá problém se ztrátou třídní knihy a může ulehčit její archivaci. Vytvoření takového informačního systému je předmětem této práce.

První část práce je věnována analýze problémové domény. Zde jsou identifikovány klíčové procesy, vytvořen doménový model a stanoveny funkční a nefunkční požadavky na systém. Další částí je rešerše, kde se autor práce zabývá již existujícími řešeními této problematiky. Následuje část věnována návrhu systému. Tato část obsahuje případy užití, návrh architektury a návrh uživatelského rozhraní několika hlavních obrazovek. Další částí je realizace, kde autor popisuje vzniklý informační systém. Jsou popsány použité technologie, představena struktura aplikace a popsán průběh testování. Závěr práce je věnován zhodnocení přínosů použití informačního systému a odhadu nákladů na vývoj a provoz systému.
\end{introduction}






